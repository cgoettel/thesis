\chapter{Introduction}\label{chp:chapter1}
Incoming freshmen struggle deciding which field they should enter. In computing, there are many fields to choose from and this can cause confusion for new students, especially because the fields are so closely related. For example, many students don't know the difference between computer science (CS), information systems (IS), and information technology (IT). It would be so nice to have a simple way to determine which field would best suit the student. But how do these fields differ? And how do these differences help us determine the right fit for incoming computing students?

One potential way to look at the differences among the various computing fields is to look at the characteristics of the students in each of these fields~--- especially how they prefer to learn. CS, IS, and IT all focus on different areas of computing and each requires a different skill set. It seems that people in these various fields even have a preference for being taught differently. Is it possible to predict in which computing discipline an incoming freshman would succeed based on their learning style preferences? Previous research has shown a correlation between learning preference and academic success for engineering students and other students, but does this correlation also exist for computing students?

In the early 1970s, Dr.\ David Kolb developed a cognitive model to represent learning preferences. His model works on a two-axis system: concrete experience (CE) versus abstract conceptualization (AC), and reflective observation (RO) versus active experimentation (AE). This two-axis spectrum is meant to describe a student's learning preferences and strengths.

The \textit{x}-axis, AE-RO, differentiates between students who prefer to learn by doing, being active, seeing results, and those who prefer to learn by watching, listening, taking their time, and relying on observations. The \textit{y}-axis, AC-CE, differentiates between students who prefer to learn by thinking things out, reasoning, being rational, and those who are more intuitive and prefer to trust their feelings.

It doesn't appear that any research has been done in this area, \textit{viz}.: examining if a student's cognitive learning preference is a factor in which computing discipline they should study. Some of the research is close, but most deal with programming aptitude or success in a first year program. Interestingly, hardly any research has taken a cognitive approach. Finally, no research was found that focused on the differences between CS, IS, and IT.

\section{Purpose}
The purpose of this research is to discover if there is a correlation between a student's preference for AC-CE and AE-RO and their GPA, and overall satisfaction in CS, IS, and IT. This research matters because incoming freshman interested in computing struggle to decide between the various, computing majors. If there is a statistically significant correlation between Kolb's learning styles and success in CS, IS, and IT, then advisement centers could use the LSI to help incoming students choose among computing majors.

\section{Research questions}
\begin{itemize}
  \item How strong is the correlation between AC-CE and AE-RO, and major GPA among CS, IS, and IT students?
  \item How strong is the correlation between AC-CE and AE-RO, and student satisfaction among CS, IS, and IT students?
  \item Is there a correlation between major GPA and student satisfaction?
  \item What is the best multiple regression model to fit these correlations?
\end{itemize}

\section{The computing disciplines}
The Association for Computing Machinery (ACM) has defined five disciplines in computing\citep{shackelford2006}: computer engineering, computer science, information systems, information technology, and software engineering. Although there is overlap between disciplines, each discipline fills its own niche. Computer engineering is focused on designing and building computer hardware and its associated software. Computer science creates low-level software, and is also concerned with the theoretical principles of computing. Software engineering is primarily focused on creating highly reliable software systems. Information technology solves general computer problems and fulfills the organizational need to integrate systems. Information systems also fulfills an organizational need, but mostly from the management side.

\section{What is cognition?}
David Kolb created the Experiential Learning Theory (ELT)\citep{kolb2005a} and used this theory as the basis for his Learning Style Inventory (LSI). This theory has its roots in famous cognitivists and learning philosophers like Dewey, Piaget, Jung, Freire, and Carl Rogers.

Kolb said that ``[l]earning is best conceived as a process, not in terms of outcomes''\citep{kolb2005a}. In this view, learning is more about the journey than the outcome. This is important because the ELT focuses on how ``[l]earning is the process of creating knowledge''\citep{kolb2005a}. This view is based on the Constructivist Theory of learning which says that a learner must construct new knowledge, or as Kolb said, ``[S]ocial knowledge is created and recreated in the personal knowledge of the learner''\citep{kolb2005b}. It's contrasted with the Transmission Model whereby ``pre-existing fixed ideas are transmitted to the learner''\citep{kolb2005a}.

This might seem like a purely semantic difference, but there's an important distinction between constructivism and transmission, much like there's an important distinction between learning as a process and learning as the end result. This is important because the ELT and this research focus on learning as a process: it matters \textit{how} students learn, not simply \textit{what} they learn.

\section{What is satisfaction?}
Satisfaction is how pleased a student is with their major decision. In order to be quantified, satisfaction was rated by the Academic Major Satisfaction Scale which was developed and validated in 2007 by Margaret Nuata. It contained questions that asked the student to rate how happy they were with their major, if they considered changing majors, and how they felt about their major:
\begin{enumerate}
  \item I often wish I hadn't gotten into this major.
  \item I wish I was happier with my choice of an academic major.
  \item I am strongly considering changing to another major.
  \item Overall, I am happy with the major I've chosen.
  \item I feel good about the major I've selected.
  \item I would like to talk to someone about changing my major.
\end{enumerate}

\section{Delimitations}
Since IT is most closely related to CS and IS, and since BYU doesn't have an SE program, the study was limited to CS, IS, and IT.

This research did not look at the socioeconomic backgrounds of the students involved. It did not consider any social pressure students may receive to join a particular field. It also did not study any of the demographics of the students.

This research was limited to seniors in CS, IS, and IT at BYU because there are too many confounding variables to properly deal with other schools and their admissions processes in a study of this size. Additionally, seniors have been exposed to myriad professors and courses, giving them a well-rounded view of the institution.
